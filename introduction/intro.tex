% pdflatex vgp_ev.tex 
\documentclass{beamer}
\usepackage[latin1]{inputenc}
\usepackage{graphicx}
\usepackage{beamerthemesplit}
\usepackage{multicol,pgf,tikz,amsmath}
\usetikzlibrary{shapes,arrows}
\usepackage{multimedia}
\usepackage{color}
%\input{epsf}

\definecolor{durham}{cmyk}{0.5,0.8,0.,0.6}
%\setbeamercovered{transparent}
\mode<presentation>
{  \usetheme[height=7mm]{Rochester}
  \usecolortheme[named=durham]{structure}
  \useinnertheme[shadow]{rounded}
  \usefonttheme[onlymath]{serif}
  %\setbeamercovered{transparent}
  \setbeamertemplate{blocks}[rounded][shadow=true]
  \setbeamertemplate{navigation symbols}{} 
  %\setbeamertemplate{footline}{\hspace{12cm} \insertframenumber/\inserttotalframenumber
  \setbeamertemplate{footline}{V. Gonzalez-Perez \hspace{10.2cm} \insertframenumber/\inserttotalframenumber }
}

%%%%%%%%%%%%%%%%
\newcommand{\lcdm}{$\rm{\Lambda CDM}$}
\newcommand{\gl}{\textsc{galform}}
\newcommand{\eg}{\textsc{eagle}}
\newcommand{\egdm}{\textsc{eagleDMO}}
\newcommand{\lgl}{\textsc{l-galaxies}}
\newcommand{\subf}{\textsc{subfind}}
\newcommand{\msun}{{\rm M}_{\odot}}
\newcommand{\mb}{M_{\rm Break}}
\newcommand{\mth}{{\rm M}_{200}^{\rm crit}}
\newcommand{\mtry}{\todo[\textcolor{blue}{Upd.}]}
\newcommand{\stry}{\todo[\textcolor{blue}{From Shaun: Upd.}]}
\newcommand{\ptry}{\todo[\textcolor{blue}{From Peter: Upd.}]}
\newcommand{\nm}{$\langle N \rangle_{M}$}
%%%%%%%%%%%%%%%%

\title{Introduction to galaxy formation: \\ Evolution of large scale structures and galaxy properties.
}
\author{{\bf Violeta Gonzalez-Perez}}
\institute{@violegp
}
\date{}


\begin{document}
\frame{\titlepage 
\vspace{-0.5cm}
\begin{center}
\includegraphics[height=0.25\textheight]{/home/violeta/charlas/fig/logos/icg-logo.pdf}
\end{center}
}


%%_____Flow charts_____________________________________
\tikzstyle{decision} = [diamond, draw=green!50!black!50,fill=green!50!black!50, text centered, text width=3.8em]
\tikzstyle{up} = [rectangle, draw, very thick,
draw=blue!50!black!50,level distance=4cm,
    text width=15em, text centered, rounded corners, minimum height=8em]
\tikzstyle{down} = [rectangle, draw, very thick,
draw=red!50!black!50,level distance=4cm,
    text width=5em, text centered, rounded corners, minimum height=3em]
\tikzstyle{up1} = [rectangle, draw, very thick,
draw=blue!50!black!50,level distance=4cm,
    text width=5em, text centered, rounded corners, minimum height=5em]
\tikzstyle{down1} = [rectangle, draw, very thick,
draw=red!50!black!50,level distance=4cm,
    text width=5em, text centered, rounded corners, minimum height=5em]
\tikzstyle{up2} = [rectangle, draw, very thick,
draw=blue!50!black!50,level distance=4cm,
    text width=10em, text centered, rounded corners, minimum height=8em]
\tikzstyle{sed} = [rectangle, draw, very thick,
draw=green!50!black!50,level distance=4cm,
    text width=5em, text centered, rounded corners, minimum height=7em]
\tikzstyle{thing} = [rectangle, draw, very thick,
draw=red!50!black!50,level distance=4cm,
    text width=5.5em, text centered, rounded corners, minimum height=2em]
\tikzstyle{line} = [draw, -latex]
\tikzstyle{cloud} = [draw, ellipse,fill=black]   
\tikzstyle{newm} = [draw=green!50!black!50, ellipse,fill=green!50!black!50,text width=4em,text centered]   
\tikzstyle{prop} = [circle,fill=red, level distance=1cm,text width=5em,text centered]   
\tikzstyle{empty} = [minimum height=2em]   


%%_____Introduction______________________________________
\begin{frame}{Modelling galaxies}
\begin{columns}
\begin{column}{7cm}
\begin{center}
   \includegraphics[width=1\textwidth]{/home/violeta/charlas/fig/intro/hst.jpg}
\end{center}
\end{column}
\begin{column}{6cm}
\begin{itemize}
\item What is a galaxy?
\item Why do we want to model galaxies?
\end{itemize}
\end{column}
\end{columns}
\end{frame}


\begin{frame}{Galaxies in a cosmological context}
\begin{tikzpicture}[node distance = 6cm, auto]
   \node [empty, rotate=90, xshift=1cm ] (galaxies) {\includegraphics[width=0.6\textwidth]{/home/violeta/charlas/fig/intro/mice_hst_big_vw1.jpg}} ; 
   \node [empty, right of=galaxies] (lss) {\includegraphics[width=0.5\textwidth]{/home/violeta/charlas/fig/eBOSS/eboss.jpg}} ; 

    % Draw edges
    \path [line, thick] (galaxies) -> (lss); 
\end{tikzpicture}

From galaxy scales (tens kpc), to the local group size (Mpcs) to the size of the observable Universe (tens of Gpc), however subpc processes affect properties at galactic scales. 

\end{frame}

\begin{frame}{Level of detail}

In order to model galaxies in a cosmological context we need to cover from galaxy scales (tens kpc), to the local group size (Mpcs) to the size of the observable Universe (tens of Gpc), however subpc processes affect properties at galactic scales. 

The level of detail in each stage depends on the problem we want to adress as resources are finite. 

I'll focus on having a sample of model galaxies large enough to be representative of the observable Universe (about 10$^{10}$ galaxies).

{\color{red} Add numbers from Millennium and Eagle and the extrapolations}
\end{frame}

\begin{frame}{How to proceed?}
\begin{itemize}
\item<1->{ How do we make a cake?
\begin{tikzpicture}[node distance = 4cm, auto]
    \node [thing] (ingredients) {Ingredients}; 
    \node [thing, right of=ingredients] (recipes) {Recipe}; 
    \node [thing, right of=recipes] (taste) {Quality check}; 

    % Draw edges
    \path [line, thick] (ingredients) -> (recipes); \path [line, thick] (recipes) -> (taste);
\end{tikzpicture}
}
\item<2->{ How do we make a galaxy?
\begin{tikzpicture}[node distance = 4cm, auto]
    \node [thing] (ingredients) {Components}; 
    \node [thing, right of=ingredients] (recipes) {Relevant \\ physical \\processes}; 
    \node [thing, right of=recipes] (taste) {Comparisson \\to observations}; 

    % Draw edges
    \path [line, thick] (ingredients) -> (recipes); \path [line, thick] (recipes) -> (taste);
\end{tikzpicture}
}
\end{itemize}
\end{frame}


\begin{frame}{Overview: Components for modelling a galaxy}
\begin{center}
\includegraphics[width=1.\textwidth]{/home/violeta/charlas/fig/intro/piechart.jpg} 
\end{center}
%\begin{itemize}
%\item What are the ingredients we need to model galaxies?
%\item Cosmology: What are the main components of the energy-matter composition of the Universe?
%\item Dark Matter(DM): how do we model it? (We'll return to this point later)
%\end{itemize}
\end{frame}

\begin{frame}{Overview: Relevant physical processes}
\begin{itemize}
\item<1-> Gravity is the main driver for the growth of structures.
\item<2-> Baryon physics: cooling of gas, flows of gas, star formation, etc.
\end{itemize}
\begin{overprint}
\begin{center}
\includegraphics<1>[width=0.9\textwidth]{/home/violeta/charlas/fig/intro/Planck_CMB_Mollweide_4k.jpg}

\includegraphics<2>[width=0.9\textwidth]{/home/violeta/charlas/fig/intro/fumagalli14.png}

{\tiny \color{orange}Planck+15}
{\tiny \color{orange} Fumagalli+15} 
\end{center}
\end{overprint}
\end{frame}

\begin{frame}{Overview: Comparisson to observations}
The fundamental observable when studying stars or galaxies is the light they emit measured by telescopes within a reduced wavelength range: 
\begin{itemize}
\item {\bf Luminosity} Energy emitted within a finite waveband per unit time. L$_{\odot}= 3.846\cdot 10^{33}{\rm erg}s^{-1}$.
\end{itemize}
\end{frame}

\begin{frame}{From luminosities to flux}
\begin{itemize}
\item {\bf Flux} Total amount of energy that crosses a unit area per unit time.
\begin{equation*}
\nu F_{\nu} = \frac{\nu_eL_{\nu_e}}{4\pi D_L^2(z)}
\end{equation*}
From the definition of redshift we have:
\begin{equation*}
1 + z = \frac{\lambda_{\rm observed}}{\lambda_{\rm emitted}}=\frac{\nu_{\rm emitted}}{\nu_{\rm observed}}=\frac{\nu_{\rm e}}{\nu}
\end{equation*}
Thus:
\begin{equation*}
F_{\nu} = (1+z)\frac{L_{(1+z)\nu}}{4\pi D_L^2(z)} 
\end{equation*}
\end{itemize}
\end{frame}


\begin{frame}{From flux to magnitudes}
\begin{itemize}
\item {\bf Magnitude}
\begin{equation*}
m_1 - m_{ref} = -2.5log_{10}\left(\frac{{\rm Flux}_1}{{\rm Flux}_{ref}}\right)
\end{equation*}
\item {\bf Absolute magnitude, M:} The magnitude of an object placed 10pc from the observer. 
\item {\bf Apparent magnitude, m:} $m = M + DM + K_e$
\begin{align*}
m - M &= -2.5log_{10}\left(\frac{F_{\nu}}{F_{\nu,{10\rm pc}}}\right) = \\
&= -2.5log_{10}\left((1+z)\frac{L_{(1+z)\nu}}{4\pi D_L^2(z)}\frac{4\pi (10{\rm pc})^2}{L_{\nu}}\right) \\
&= 5log_{10}\left(\frac{D_L(z)}{(10{\rm pc})}\right)  - 2.5log_{10}\left((1+z)\frac{L_{\nu_e}}{L_{\nu}}\right) 
\end{align*}
\end{itemize}
\end{frame}

\section{The luminosity function}
\begin{frame}{Statistical approach to galaxies}
\begin{itemize}
\item Probability of finding one galaxy in a diferential volume: $dP=n\,dV$, with n as the mean number of objects in a finite volume, V, per unit volume.
\item The average total number of objects will be $<N>=nV$.
\item Probability of finding a galaxy within $dV$ with a mass, $M_*-\frac{dM_*}{2}\le M_* \le M_*+\frac{dM_*}{2}$: \begin{equation*} dP = \Phi(M_*,t)dM_*dV \end{equation*}
\item We refer to $\Phi(M_*,t)$ as the stellar mass function, i.e., the average number of galaxies per stellar mass bin per volume at a given time (or redshift), such that, \begin{equation*} n = \int_{-\infty}^{\infty}\Phi(M_*,t)dM_* \end{equation*}
\end{itemize}
\end{frame}

\begin{frame}{The luminosity function}
\begin{center}
\includegraphics[height=1.2\textheight]{/home/violeta/charlas/fig/lf_sps/B_z0.pdf}
\end{center}
\end{frame}


\begin{frame}{The observed mass/luminosity function}
Typical Schechter function in terms of L and M.

How do we go from L to M? Assumptions needed to get stellar masses from luminosities
\end{frame}


%_______Merger trees and assumtions______________

\begin{frame}{How to proceed depends on your goal:}
\begin{flushleft}
\includegraphics[width=1\textwidth]{/home/violeta/charlas/fig/intro/sizes_matthieu.png}
\end{flushleft}

\vspace{1cm}Example: 

The final Eagle run took $4.5\cdot 10^{6}$ CPU hours 

+ $40\cdot 10^{6}$ CPU hours for calibration

\begin{center}
{\tiny {\sc credit:} Matthieu Schaller}
\end{center}
\end{frame}



\begin{frame}{Modelling DM:}

{\color{blue} a) N-body simulations:} Softening length and volume are the limiting factors.

{\color{blue} b) Approximattive methods:} (LPT, COLA, PINOCHIO, etc) They are calibrated in simulations, but they are useful for covariance matrices.
\end{frame}


\begin{frame}{Modelling galaxies:}

{\color{blue} a) In parallel:}  hydrodynamical simulations
\begin{center}
   \includegraphics[height=0.4\textheight]{/home/violeta/charlas/fig/eagle/eagle.png}
\end{center}
{\color{blue} b) In series:}  SAMs, EMs, SHAMs, HOD modelling

   1.\includegraphics[height=0.27\textheight]{/home/violeta/charlas/fig/ms_z0.jpg}{ } 
    2.\includegraphics[height=0.27\textheight]{/home/violeta/charlas/fig/intro/mergertree.jpg}{ }
    3.\includegraphics[height=0.27\textheight]{/home/violeta/charlas/fig/galform_fake_panstarrs.jpeg}
\end{frame}

\begin{frame}{Simulated volumes:}
\begin{center}
\includegraphics[width=1.\textwidth]{/home/violeta/charlas/fig/intro/sizes_clau.png}

{\tiny {\sc credit:} Claudia Lagos}
\end{center}
\end{frame}


\begin{frame}{The modelling approach depends on the scientific question}
\begin{center}
\includegraphics[height=.9\textheight]{/home/violeta/charlas/fig/intro/Lagos.pdf}

{\tiny {\sc credit:} Claudia Lagos}
\end{center}
\end{frame}


\begin{frame}{Focusing on model galaxies calibrated at $z=0$}
Many studies attempt to have statistically representative galaxies at $z=0$ and study then how they evolve. 
\begin{center}
\includegraphics[width=1\textwidth]{/home/violeta/charlas/fig/schaye15.png}

{\tiny \color{orange} Schaye+15}
\end{center}

Other approaches are needed for other purposes: {\tiny have a catalogue of model galaxies that reproduce the spatial distribution of galaxies at a given z, produce a sample of galaxies with the same observed colours as LBG at $z=3$, to study their evolution, etc.}
\end{frame}


\begin{frame}{The dark matter merger trees}
\begin{center}
\begin{tikzpicture}[node distance = 2cm, auto]
    \node [cloud] (cosmo) {\textcolor{white}{$\Lambda$CDM   Cosmology}}; 
    \node [cloud, below of=cosmo, yshift=1cm] (dm) {\textcolor{white}{DM Merger trees}}; 
    \node [empty, below of=dm, yshift=-1.5cm] (lee) {\includegraphics[height=0.7\textheight]{/home/violeta/charlas/fig/lee14_smf.png}};
    \node [empty, right of=lee, yshift=1cm, xshift=4cm] (sch) {\includegraphics[height=0.65\textheight]{/home/violeta/charlas/fig/schaller15.png}};
    \node [empty, below of=sch, yshift=-1.5cm,xshift=1cm] (text) {\tiny {\sc credits:} Lee+14, Schaller+15};

    \path [line, thick] (cosmo) -> (dm); \path [line, thick] (dm) -> (lee);
\end{tikzpicture}
\end{center}
\end{frame}


\begin{frame}{Basic steps for modelling galaxies}
\begin{tikzpicture}[node distance = 2cm, auto]
    \node [cloud] (dm) {\textcolor{white}{Gas bounded to a gravitation potential}};
    \node [decision, below of=dm] (cool) {\textcolor{white}{Efficient cooling?}}; 
   \node [empty, left of=cool, xshift=0.4cm, yshift=0.2cm] (nocool) {\textcolor{green!50!black!50}{No}};
   \node [empty, below of=cool, xshift=-1.4cm, yshift=1.1cm, rotate=40] (yescool) {\textcolor{green!50!black!50}{Yes}};
    \node [thing, left of=cool, xshift=-1cm] (dh) {Dark Halo}; 
    \node [empty, below of=dh, yshift=-1cm, xshift=-0.5cm] (sun) {\includegraphics[height=0.35\textheight]{/home/violeta/charlas/fig/intro/sun.png}}; 
    \node [decision, below of=cool, yshift=-1cm] (jm) {\textcolor{white}{Large j?}}; 
   \node [empty, right of=jm, yshift=1cm, xshift=-1.2cm, rotate=40 ] (yesj) {\textcolor{green!50!black!50}{Yes}};
   \node [empty, right of=jm, yshift=-1cm, xshift=-1.2cm, rotate=-40] (noj) {\textcolor{green!50!black!50}{No}};
    \node [thing, right of=jm, yshift=1.5cm] (disc) {Disc}; 
    \node [thing, right of=jm, yshift=-1.5cm] (bulge) {Bulge}; 
    \node [decision, right of=jm,xshift=1.5cm] (merge) {\small \textcolor{white}{Merger? Disc instability?}}; 
    % Draw edges
    \path [line, thick] (dm) -> (cool);
    \path [line, thick] (cool) -> (dh); \path [line, thick] (dh) -> (dm);
    \path [line, thick] (cool) -> (sun); \path [line, thick] (sun) -> (jm);
    \path [line, thick] (jm) -> (disc);
    \path [line, thick] (disc) -> (merge); \path [line, thick] (jm) -> (bulge);
    \path [line, thick] (merge) -> (bulge); 
\end{tikzpicture}
\end{frame}



\begin{frame}{Choices within galaxy models}
\begin{tikzpicture}[node distance = 2cm, auto]
    % Place nodes
    \node [empty,xshift=-2cm] (galform) {\includegraphics[height=0.15\textheight]{/home/violeta/charlas/fig/galform.jpg}};
   \node [up, right of=galform,xshift=3.5cm,yshift=1cm] (disk) {\small \textcolor{durham}{ Beisdes the free parameters that need calibration, some choices need to be make: 
\hspace{-0.5cm}{\begin{itemize}
\item Chemical Evolution:\\ IRA?, yield? R?
\item IMF
\item Stellar population 
\item Dust attenuation model
\end{itemize}}
}};
    \node [cloud, above of=galform] (dm) {\textcolor{white}{DM Merger trees}}; 
    \node [cloud, above of=dm,yshift=-1cm] (cosmo) {\textcolor{white}{$\Lambda$CDM   Cosmology}}; 
    \node [prop, below of=galform,yshift=-1.5cm] (gprop) {\textcolor{white}{Observable galaxy  properties}}; 
   \node [decision, right of=gprop,xshift=2cm] (update) {\small \textcolor{white}{Fit obs. \\ at z=0?}}; 
   \node [newm, right of=update,xshift=1.5cm] (sam) {\textcolor{white}{Stable model}}; 

   \node [empty, below of=galform, xshift=3cm, yshift=-0.3cm, rotate=-40] (no) {\textcolor{green!50!black!50}{NO}};
   \node [empty, below of=galform, xshift=1.8cm, yshift=0.2cm,rotate=-40] (nom) {\textcolor{green!50!black!50}{\tiny update free parameters}};
   \node [empty, left of=sam, yshift=0.2cm, xshift=0.2cm] (yes) {\textcolor{green!50!black!50}{YES}};
   \node [empty, left of=update, yshift=0.2cm] (test) {\textcolor{green!50!black!50}{Test}};
%
%%
%%
    % Draw edges
    \path [line, thick] (cosmo) -> (dm);
    \path [line, thick] (dm) -> (galform);
    \path [line, thick] (galform) -> (gprop);
    \path [line, thick,draw=green!50!black!50] (gprop) -> (update);
    \path [line, thick,draw=green!50!black!50] (update) -> (galform);
    \path [line, thick,draw=green!50!black!50] (update) -> (sam);
%%    \path [line] (disks) -> (prop);
%%    \path [line] (prop) -> (update);
%
\end{tikzpicture}
\end{frame}


%-----------------------------------------------
\end{document}
